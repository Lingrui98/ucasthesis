\chapter{分支预测背景知识}\label{appendix:background}
本章节简要介绍本设计中用到的分支预测部件的背景知识。
\section{TAGE分支方向预测器}
TAGE分支方向预测器的主要任务有三项:预测、更新和回滚。作为通过历史信息决定预测结果的部件,它的主要存储开销用来维护M+1张记录历史分支信息的表。其中有一张表是一个直接用分支指令地址索引的bimodal基预测器,而另外M张表分别使用不同长度的分支历史和分支地址进行哈希索引,这些分支历史的长度依次形成几何级数。每一张表采用了类似cache的带标签的存储方式,用前述的哈希值查找得到一个表项后,将其中的tag与PC和分支历史的另一种哈希值进行匹配。在预测时,所有的表会进行并行查找,仅当某张表的tag匹配成功时,它才能提供候选预测值。最终的预测结果由能提供预测结果的使用最长历史信息的表提供。如果M张表的标签都匹配失败,那么使用bimodal基预测器的预测结果。下图为M=4的TAGE分支预测器结构示例,其中bimodal表项均为两位饱和计数器,而M张带标签的表的表项由三部分组成:pred、tag、u。其中pred为提供预测结果的三位饱和计数器,它的最高有效位决定了预测方向;tag为10~13位标签;u是useful counter的简称,它是一个两位饱和计数器,会影响TAGE更新时的行为。

TAGE的更新机制是它的关键所在。将此次预测中决定最终结果的表记为provider,provider即为当次预测中标签匹配成功且使用了最长历史的表。将标签匹配成功且使用次长历史的表记为altpred。

Pred域的更新:若预测结果正确,则依预测结果更新provider中对应表项中的pred域;反之,除了以相同方法更新provider的pred域以外,假如provider并不是所有表中使用最长历史的那张表,则试图在更长历史的那些表中选择一张表分配一个表项给这条分支。具体的分配策略决定于这些表中对应表项	useful counter的值:如果所有对应表项的useful counter都大于0,则将它们都减一,此次不分配新的表项;如果存在某个u=0的表项,则它可以被分配给这条分支;如果存在至少两个u=0的表项,则以更倾向于历史长度较低表项的概率分配表项(进一步阐述)。分配后表项pred域初始化为weak correct(例如0),u域初始化为0。

Useful域的更新:当altpred的结果与provider不同时,更新provider的u域。此时如果provider的结果正确,u增1,反之u减1。Useful域实行定期flush机制,每隔256K条分支,将一排u域的MSB清零。

某次分支预测的回滚需要保存做出预测前一刻预测器内的一些信息,以及该次预测中产生的某些信息。这些信息打包后送入BRQ,以待用于回滚和更新。