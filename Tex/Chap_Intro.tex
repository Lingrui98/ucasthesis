\chapter{引言}\label{chap:intro}

\section{研究背景}

随着芯片制程的提高,现有工艺逐渐逼近物理极限,摩尔定律和丹纳德缩放比例定律走向终结。Patterson和Hennessy在图灵演讲中指出,体系结构领域正走向变革的黄金时代\cite{hennessy2019new},人们不再能仅依靠制程的进步逐年在芯片上实现更复杂的架构。他们认为领域专用架构是接下来芯片性能提升的重点。然而,随着制程进步,芯片复杂度变高,从而导致芯片的设计成本不断攀升,这阻碍了创新性的想法和设计付诸实践。

针对这个问题,软件行业已经有了开源作为答案。开源的意义在于成果共享,避免重复劳动,提升行业整体效率。以Linux为核心的开源软件栈,大大减少了开发者去开发一款新应用的时间和人力投入,有了诸如GCC之类的开源实现,用户只需要复用这些软件和库,需要自己实现的逻辑大大减少。同时开源和商业也并不冲突,选择适当的开源协议,可以让开源的成果为商业所用。RedHat就是Linux商业化的最著名的例子。

受到软件行业的启发,硬件行业也引入了开源的概念,首先诞生了一款设计理念先进并且开放授权的指令集设计RISC-V\cite{asanovic2014instruction}。它自从诞生以来在工业界和学术界的影响力与日俱增。在此基础上,一款成功的、高性能的RISC-V处理器核实现,不论对于工业界还是学术界都至关重要。这样的实现可以降低工业界IP的授权成本和开发周期;而学术界可以将其作为一个比软件模拟器更精准的体系结构探索平台。理想情况下,它可能成为开源芯片领域类似Linux的存在。而一些新的硬件构造语言例如Chisel\cite{bachrach2012chisel}、SpinalHDL等,将编程语言的高级特性与硬件设计相结合,进一步提高了硬件代码的表达能力,从而提升了芯片敏捷开发的可能性。

\section{研究动机}
\subsection{现有分支预测部件设计的不足}
尽管目前有一些开源的分支预测部件设计,但它们都没有同时做到高预测和高主频。分支方向预测器对于分支预测部件至关重要,目前state-of-the-art级别的分支方向预测器主要有两种:TAGE\cite{seznec2006case}和perceptron\cite{jimenez2001dynamic}。当前开源高性能处理器核的实现中,在分支预测器部分实现了两者之一的只有加州伯克利的SonicBOOM\cite{zhao2020sonicboom},但它并没有经过流片验证。经我们评估,其主频只能达到不到1GHz。因此,目前并没有符合要求的高性能分支预测部件设计。

\subsection{分支预测部件设计的挑战}
针对现有开源分支预测部件设计的补足,本工作旨在实现同时达到高预测性能和高主频的分支预测部件。

对于预测性能方面,主要有如下的评价指标:预测宽度、分支时延和预测准确率。下面将会对它们进行逐点的分析:
\begin{itemize}
    \item 预测宽度需要配合取指宽度,取指宽度又与后端流水线宽度息息相关,前端的取指宽度需要保证后端的指令供给;
    \item 现代处理器主频逐渐提高,与此同时分支预测策略也渐趋复杂,导致难以在愈发有限的时钟周期内完成复杂的分支预测任务\cite{jimenez2000impact}。工艺制程上的进步并不能解决两者的冲突,以TAGE分支方向预测器\cite{seznec2006case}为例,它的预测流程至少要分为历史折叠、存储表访问和选择最长匹配三个阶段,这三个阶段都需要一个时钟周期完成。在很宽的取指宽度下,大概率每一拍都会遇到分支指令。如果每一条分支指令都要等到数个时钟周期后出结果,那么最终的指令供给将无法达到处理器后端的要求;
    \item 前两个指标保证了前端能给后端足够的指令供给,而预测准确率保证了这些指令供给的有效性。如果误预测率很高,前端取到的很多指令都在错误的执行路径上,最终会被冲刷掉。即使供指带宽很高,处理器得到的真实指令供给也会不足,导致整体性能不高。
\end{itemize}

在预测性能一定的前提下,整体性能和频率成正比,高主频对于总体性能至关重要。本设计的基本目标是让分支预测部件的主频达到1.5GHz,避免成为整个处理器核的频率瓶颈。

\section{论文的主要内容与贡献}
针对前述指标和挑战,本工作在参考已有实现的基础上,在香山开源RISC-V处理器核中实现了一款高性能分支预测部件,初步解决了这些问题。

我们采用全并行的预测方法实现高预测宽度,达到了最大16条指令/周期的预测宽度;用覆盖预测架构解决主预测器长分支时延的问题,实现了最佳情况下无空泡的取指;通过优化主预测器的误预测率,降低分支预测部件的整体误预测率,在SPEC CPU 2006\cite{henning2006spec}上达到了4.17的MPKI;最后,通过一系列细节上的优化,使频率达到了1.5GHz。

\section{论文组织}
本文的后续章节组织结构如下:

第\ref{chap:design}章介绍了香山处理器分支预测部件的设计,包括其整体预测架构、一些子预测部件的设计以及在此架构设计下的预测流程。

第\ref{chap:impl}章介绍了香山处理器分支预测部件的实现,包括本工作的代码基础和所选用的语言,也介绍了实现上一些关键问题的处理方案,包括分支历史的管理、跨取指包的RVI指令的预测以及预测器更新的问题。

第\ref{chap:eval}章介绍了香山处理器分支预测部件的验证和评估工作,其中评估部分包括后端评估和预测性能评估。

第\ref{chap:result}章对本文工作进行总结,并提出了接下来可能的研究方向。