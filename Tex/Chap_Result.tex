\chapter{结论和展望}\label{chap:result}
本工作的目的是设计实现一款高性能的分支预测部件,满足香山RISC-V高性能处理器核的指令供给,以进一步支持香山处理器的流片。本工作在设计实现上进行了一些优化,在频率和性能方面都有提升。在主预测器上将SonicBOOM\cite{zhao2020sonicboom}分支预测部件中的L-TAGE预测器\cite{seznec2011new}进一步用Statistical Corrector\cite{seznec2014tage}加强。我们在仿真器和模拟器上进行性能对比评估表明,本工作的实现基本符合设计预期。后端评估结果表明,本工作在TSMC 28nm工艺下,频率能达到1.5GHz,相对SonicBOOM同规格配置下的分支预测器频率近乎提升一倍。

本工作实现的分支预测部件将随香山高性能处理器发布,源码也将一并开源,推动整个开源芯片领域的发展。

未来的工作可以从如下方向进行:
\begin{enumerate}
    \item 目前的工作中尚未包括间接跳转预测器的实现,而现代程序设计语言中例如虚函数表的实现会在程序中大量引入间接跳转指令。一个完善的间接跳转预测器将会在这方面减少大量的误预测,提升处理器的整体性能;
    \item 目前实现的主分支预测器是原设计的一种简化实现,具体表现在如下方面:
    \begin{itemize}
        \item 所使用的分支历史长度较短,历史表不多;
        \item 未实现Immediate Update Mimicker\cite{seznec2011new}
    \end{itemize}
    针对上述两个方面进行进一步优化后,预计分支误预测率可以进一步降低;
    \item 目前的时序只能满足1.5GHz,高性能设计要求更高的主频,因此需要对时序进行进一步优化。
\end{enumerate}