\chapter{分支预测部件微架构设计}\label{chap:design}
本章我们详细介绍香山处理器核分支预测部件的微架构设计。首先我们对于前述影响性能的各项指标,描述我们针对性的设计思路。接着详细介绍分支预测部件中的模块及每个模块的功能。

分支预测部件和取指部件关联紧密,它们之间的关系主要有两类:
\begin{enumerate}
    \item 紧耦合:取指和预测流水完全同步;
    \item 解耦\cite{reinman1999scalable}:先预测后取指,用队列连接。
\end{enumerate}

出于项目进度的考虑,我们选择紧耦合实现。
\begin{figure}[!htbp]
    \centering
    \includegraphics[width=\textwidth]{bpu}
    \caption{分支预测部件结构示意图}
    \label{fig:bpu}
\end{figure}
\section{设计思路}
图\ref{fig:bpu}展示了分支预测部件的结构。我们首先用一个小节阐述我们在前述关键指标驱动下的设计思路。
\subsection{并行预测以提高预测宽度}
香山处理器核的后端是六发射结构,理想情况下每拍能消耗前端取指单元提供的六条指令。为了满足后端的指令供给,取指宽度至少不小于六。我们为了留出一些余量,将取指宽度设置为32字节,以标准RVI指令计算,每拍最多取出八条指令。由于香山处理器核实现了RISC-V压缩指令扩展,指令大小可能为二或四字节,所以每拍实际上能取出最多十六条指令(假如全部是压缩指令)。在这样的取指宽度下,每一次取指遇到一条甚至多条转移指令的概率很大,因此我们需要支持多条指令的并行预测以避免行内指令的结构相关带来的性能损失。香山处理器核分支预测部件采用了全并行预测的设计,对于一次取到的32字节中的全部16个可能的指令起始位置都进行预测。
\subsection{流水化覆盖多级预测以降低分支时延}
我们实现的主分支方向预测器的访问延迟是三个时钟周期,为了缓解分支时延,我们首先将各分支预测部件流水化,让它们在流水线后端不阻塞的情况下,每一周期都能处理一个新的取指请求。如果仅是这样,对于一条分支指令,它仍需要三个时钟周期才能得到预测结果,而与其相邻的下一个取指请求依赖于这条分支指令的预测结果。为了达到不阻塞流水线的目的,简单的处理办法是在分支预测结果尚未得到时,默认其不跳转。于是对于实际执行结果不跳转的分支指令,相当于没有流水线空泡。但对于实际跳转的分支指令,则需要在得到预测器结果时,向之前的流水级发送重定向请求,执行流水线冲刷,这样仍存在流水线的空泡。为了解决这个问题,我们在设计上引入多级覆盖预测,额外使用一些时延较小的简单的分支预测部件,很快地得到一个准确率相对较低的预测结果,用于快速发出下一个取指请求。当主预测器的预测结果与简单预测器不符时,再冲刷流水线,并根据主预测器的预测结果重新发出取指请求。这样能够进一步将跳转分支的预测时延限制在简单预测器和主预测器结果不符的情况下。
\subsection{使用TAGE-SC-L提高分支方向预测准确率}
% 程序依赖分支指令构成控制流,现代程序平均每五条就有一条分支指令,因此分支指令的准确预测对处理器总体性能至关重要。分支指令的执行有两个基本要素:跳转方向和目标地址,分支预测需要同时解决这两个问题。对于RISC-V指令集架构来说,条件分支属于直接跳转,其目标地址由指令地址加上指令码中指定的偏移计算,因此一般情况下是不会改变的,比较容易预测。而条件分支的跳转方向会随程序行为变化而变化,属于分支预测的重点。目前业界普遍的做法是全局分支历史信息在逻辑上可以用一个移位寄存器实现,顺序记录了到当前周期为止,执行过的前n条分支指令的跳转方向。每当一条分支指令被执行,便在移位寄存器中移入一位,并移出最老的一条分支的信息。
为了提高分支方向预测准确率,我们引入了TAGE-SC-L预测器\cite{seznec2014tage}作为分支方向的主预测器。TAGE-SC-L是目前学术界state-of-the-art的分支预测方法。它主要利用长度呈几何级数的全局分支历史信息经异或折叠后与程序计数器PC的值哈希,分别用于索引多个表中的项,并对于每个表计算一个Tag,用于判断是否匹配,其预测结果由匹配到的使用最长历史长度的那个表决定。TAGE-SC-L的具体算法较为复杂,此处不再赘述。

\section{分支预测架构}
本节详细介绍分支预测部件的架构。如图\ref{fig:bpu}所示,分支预测部件主要由各子预测部件和顶层逻辑组成,另外还包括实现于取指部件中的分支历史管理逻辑。

各子预测部件包括:
\begin{itemize}
    \item MicroBTB(uBTB):分支目标微缓冲,提供一周期内的分支方向和跳转目标预测;
    \item BTB:分支目标微缓冲,提供两周期内的跳转目标地址预测;
    \item BIM:作为BTB的补充,在两周期内用两位饱和计数器提供分支方向预测;
    \item TAGE-SC-L:主分支方向预测器,在三周期内提供分支方向的准确预测;
    \item RAS:返回地址栈,在最后一周期为返回指令提供目标地址预测。
\end{itemize}

顶层逻辑主要将各子预测部件的预测按流水级以统一的接口输出供取指模块使用,以及处理部分子预测部件的结果需要在多个流水级复用的情况。虽然分支预测部件整体是流水化的,但由于它和取指单元紧耦合,因此它只需要复用取指单元的握手信号,便可以驱动自身流水级的行进。

接下来将沿一个取指请求的预测流程,详细介绍几个子模块及其处理逻辑。每一级都会对每一条指令是否跳转,以及其在跳转情况下的目标地址进行预测。

\subsection{第一级预测:MicroBTB}
取指请求PC暂存一拍后,送入MicroBTB进行访问,从MicroBTB的全相联CAM中,根据PC高位生成的Tag判断每一条指令是否命中。如果不命中则视为不跳转(不论是否是转移指令);如果命中则根据其中存储的isBr状态位,决定是否使用对应的两位饱和计数器的结果判断条件分支方向。若isBr为1,则该指令是条件分支,用两位饱和计数器的高位作为预测的方向;反之则是无条件跳转,默认跳转。

\subsection{第二级预测:BTB和BIM}
取指请求PC经过存储映射逻辑,直接输入BTB和BIM的SRAM的读端口,在下一个时钟周期得到预测信息。对于BTB的预测信息,和MicroBTB的处理类似,根据Tag判断每一条指令的命中情况,并根据命中情况决定是否使用BIM的分支方向预测信息。这些逻辑做完后暂存一拍,在第二个流水级使用预测结果。

\subsection{第三级预测:TAGE-SC-L和RAS和预译码信息}
与取指请求对应的分支历史在暂存一拍后,根据不同表的历史长度和表项个数进行异或折叠,折叠后与暂存一拍的取指请求PC哈希后分别得到访问SRAM的索引和用于匹配的Tag。索引直接送入TAGE-SC-L的SRAM的读端口,而Tag暂存一拍,与下一个时钟周期得到的预测数据结合,经过Tag匹配和最长历史选择等逻辑后得到最终的分支方向预测,暂存一拍后用于第三级流水的分支方向预测。

RAS是一个栈结构,它的预测永远来自栈顶项,只需要在第三级流水根据预译码给出的转移指令类型信息,决定是否让某一条指令的目标地址预测选择来自RAS的结果。

预译码信息来自取指模块,根据指令Cache取出的指令码进行简单的译码操作,结果在第三个流水级送入分支预测部件使用。它提供转移指令的类型信息,以及条件分支和无条件直接跳转类型指令的目标地址。

\section{预测器的恢复和更新}\label{design:restore_update}
各个子预测部件在预测时都会产生一些预测相关的信息,它们需要被保存下来,以待将来的误预测恢复和提交后更新。当产生分支误预测,前端收到重定向请求时,有一些预测器的状态经历了推测更新,已经和错误分支刚刚预测结束时有所差别,这时候就需要用保存下的信息将预测器状态恢复到预测完目标分支之后的状态,从而保证正确路径上的分支所用的分支历史等信息都是正确的。当转移指令提交时,相应的预测信息会送回各子预测部件进行相应的表项更新。香山处理器中的预测信息存储放在了一个独立的队列中实现。当最后一级预测完成时,会将该取指请求对应的所有指令的预测信息作为一项入队;在该取指请求取到的所有指令都提交完成后,该项出队。